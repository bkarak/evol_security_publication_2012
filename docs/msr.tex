\documentclass[conference]{IEEEtran}

\usepackage{url}
\usepackage{amsmath}
\usepackage{graphicx}
\usepackage{subfigure}

\hyphenation{op-tical net-works semi-conduc-tor}

\begin{document}

\title{Dismal Code: Studying the Evolution of Security Vulnerabilities}


\author{
\IEEEauthorblockN{Dimitris Mitropoulos \and Vassilios Karakoidas \and Panos Louridas \and Diomidis Spinellis}
\IEEEauthorblockA{Department of Management Science and Technology\\
Athens University of Economics and Business\\
Email: \{dimitro, bkarak, louridas, dds\}@aueb.gr}
\and
\IEEEauthorblockN{Georgios Gousios}
\IEEEauthorblockA{Department of Software and Computer Technology\\
Delft University of Technology \\
Email: G.Gousios@tudelft.nl}
}

\maketitle

\begin{abstract}
%\boldmath
The abstract goes here.
\end{abstract}

\begin{IEEEkeywords}
Software Vulnerabilities, Static Analysis, Software Evolution, Software
Security, Maven, FindBugs.
\end{IEEEkeywords}

\IEEEpeerreviewmaketitle

\section{Introduction}

A security bug is a programming error that introduces a potentially
exploitable weakness into a computer system~\cite{SSL12}. This weakness could lead to a
security breach with unfortunate consequences in different layers, like databases,
native code, applications, libraries and others. Despite the significant
effort to detect and eliminate such bugs~\cite{SZ12}, little attention has been paid to
study them in relation to software evolution~\cite{L96, LRWPT97}. In this paper we present how we
used a large software ecosystem to analyze how evolving software packages are
related to the different types of software vulnerabilities.

One of the most common approaches to identify software vulnerabilities is
{\it static analysis}~\cite{CW07}. This kind of analysis involves the
inspection of the program's source or object code without executing
it. For our research we used {\it FindBugs},\footnote{\url{http://findbugs.sourceforge.net/}}
a popular static analysis tool that has already been used in
research~\cite{AP10, HP07}. Specifically, we ran FindBugs on all the project
versions of all the projects that exist in the Maven\footnote{\url{http://maven.apache.org/}}
repository (approximately 265GB of data). Then we observed the changes that
involved the security bugs and their characteristics. This research builds upon
our earlier work on the topic~\cite{MGS12}.

We chose to focus our study on the security bugs rather than other
software bugs because compared to the other software bugs,
security bugs have a distinct feature: they are critical. Specifically, a software bug can
lead to a mulfunction of a software application that runs under specific
requirements but a security bug can allow a malicious user to alter the
execution of the entire application. In this case, such bugs could span a wide range
of security and privacy issues, like viewing sensitive information, the destruction or
modification of sensitive data, denial of service and others.
Hence, one of the basic pursuits in every new software release should be the
mitigation of such bugs.

The main contributions of this research are:

\begin{itemize}
	\item the analysis of how security bugs evolve in time. To achieve
this, we inspect every project per version.
	\item the relation of security bugs to the component's size.
	\item bug persistense between versions.
	\item the correlation of security bugs with other bug categories.
	\item the relation of software dependencies and security bugs.
\end{itemize}

In the rest of this paper we
describe the processing of our data and our experiment (Section \ref{sec:meth}),
present and discuss the results we obtained (Section \ref{sec:res},
outline related work (Section \ref{sec:rel}),
and end up with an conclusion and directions for future work (Section \ref{sec:con}).

\section{Methodology}
\label{sec:meth}

\subsection{Data Provenance and Processing}
\label{sec:data}

The subject of our study was a snapshot of the maven repository. Before
analyzing the data of the repository we performed a number of functional fits
on the data.

First, we obtained the repository snapshot (January 2012) that was previously
used by -- et. al~\cite{}.

\begin{table}
\centering
\caption{Descriptive Statistics Measurements for the Maven Repository}
\label{tbl:repository}
\begin{tabular}{l r}
 \hline
Projects & 17,505\\
Versions (total) & 114,399\\
Min & 1\\
Max & 337\\
Mean & 6.54\\
Median & 3\\
Range & 336\\
1$^{st}$ Quartile & 1\\
3$^{rd}$ Quartile & 8\\
\hline
\end{tabular}
\end{table}

\subsection{Experiment}
\label{sec:exp}

\section{Results and Analysis}
\label{sec:res}

\section{Related Work}
\label{sec:rel}

There are noumerous methods for mining software repositories in the context
of software evolution~\cite{KCM07}. In this section we focus on the ones
that highlight the relationship between software bugs and evolution.

Cite this~\cite{ZAH11} and this~\cite{MNN11}.

\section{Conclusions and Future Work}
\label{sec:con}

Our work could (?) aid bug-prediction models.

\section*{Acknowledgments}

The project is being co-financed by the European Regional Development Fund (ERDF)
and national funds and is a part of the Operational Programme ``Competitiveness \&
Entrepreneurship" (OPCE II), Measure ``COOPERATION" (Action I).

\bibliographystyle{IEEEtran}
\bibliography{msr} 

\end{document}


