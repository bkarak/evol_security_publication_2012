\documentclass[conference]{IEEEtran}

\usepackage{url}
\usepackage{amsmath}
\usepackage{graphicx}
\usepackage{subfigure}

\hyphenation{op-tical net-works semi-conduc-tor}

\begin{document}

\title{Dismal Code: Studying the Evolution of Security Vulnerabilities}


\author{
\IEEEauthorblockN{Dimitris Mitropoulos \and Vassilios Karakoidas \and Panos Louridas \and Diomidis Spinellis}
\IEEEauthorblockA{Department of Management Science and Technology\\
Athens University of Economics and Business\\
Email: \{dimitro, bkarak, gousiosg, louridas, dds\}@aueb.gr}
\and
\IEEEauthorblockN{Georgios Gousios}
\IEEEauthorblockA{Department of Software and Computer Technology\\
Delft University of Technology \\
Email: G.Gousios@tudelft.nl}
}

\maketitle

\begin{abstract}
%\boldmath
The abstract goes here.
\end{abstract}

\begin{IEEEkeywords}
Software Vulnerabilities, Static Analysis, Software Evolution, Maven, FindBugs.
\end{IEEEkeywords}

\IEEEpeerreviewmaketitle

\section{Introduction}

A security-related bug is a programming error that introduces a potentially
exploitable weakness into a computer system. This weakness could lead to a
security breach with unfortunate consequences in different layers, like databases,
native code, applications, libraries and others. One of the most common
approaches to identify software vulnerabilities is {\it static analysis}. This
kind of analysis is performed by automated tools either on the program's source
or object code and without actually executing it.

Cite this~\cite{MNN11}.

\section{Related Work}

Cite this~\cite{ZAH11}.

\section{Methodology}
\label{sec:meth}

\subsection{Data Provenance and Processing}
\label{sec:data}

\begin{table}
\centering
\caption{Descriptive Statistics Measurements for the Maven Repository}
\label{tbl:repository}
\begin{tabular}{l r}
 \hline
Projects & 17,505\\
Versions (total) & 114,399\\
Min & 1\\
Max & 337\\
Mean & 6.54\\
Median & 3\\
Range & 336\\
1$^{st}$ Quartile & 1\\
3$^{rd}$ Quartile & 8\\
\hline
\end{tabular}
\end{table}

A snapshot of the maven repository - January 2012.

\subsection{Experiment}
\label{sec:exp}

\section{Results and Analysis}
\label{sec:res}

\section{Conclusion}
\label{sec:con}
The conclusion goes here.

\section*{Acknowledgments}

The project is being co-financed by the European Regional Development Fund (ERDF)
and national funds and is a part of the Operational Programme ``Competitiveness \&
Entrepreneurship" (OPCE II), Measure ``COOPERATION" (Action I).

\bibliographystyle{IEEEtran}
\bibliography{msr} 

\end{document}


